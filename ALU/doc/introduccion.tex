\section{Introducción}

En este informe se documenta el desarrollo de una Unidad Aritmético-Lógica (ALU) implementada en una FPGA Basys 3.  
La ALU constituye un bloque fundamental en el diseño de procesadores y sistemas digitales, ya que permite realizar operaciones aritméticas, lógicas y de desplazamiento sobre datos binarios.  

El objetivo principal de este trabajo práctico es adquirir experiencia en la descripción de hardware mediante Verilog, la validación con bancos de prueba, la simulación usando la herramienta Vivado, y la implementación física real.  

Las herramientas utilizadas para el desarrollo fueron:  
\begin{itemize}
    \item \textbf{Lenguaje Verilog} para la descripción del hardware.
    \item \textbf{Vivado} como entorno de síntesis, simulación e implementación.
    \item \textbf{FPGA Basys 3} como plataforma de ejecución del diseño.
\end{itemize}
