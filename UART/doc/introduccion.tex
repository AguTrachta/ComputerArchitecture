\section{Introducción}

La comunicación serie es uno de los mecanismos más utilizados en sistemas digitales para transmitir información de manera simple y eficiente. Dentro de este esquema, el estándar \textbf{UART} (Universal Asynchronous Receiver/Transmitter) constituye una de las soluciones más difundidas debido a su simplicidad de implementación, bajo costo y compatibilidad con una amplia variedad de dispositivos.

En este trabajo se presenta el diseño e implementación de un sistema digital basado en \textbf{UART}, desarrollado en lenguaje \textit{Verilog HDL}. El sistema no solo incluye los bloques clásicos de transmisión y recepción, sino que además incorpora:

\begin{itemize}
    \item \textbf{Un generador de baudios} para sincronizar la comunicación.
    \item \textbf{Memorias FIFO} que permiten desacoplar los datos recibidos y transmitidos.
    \item \textbf{Una Unidad Aritmético-Lógica (ALU)} encargada de procesar los operandos recibidos.
    \item \textbf{Un bloque de interfaz} que organiza los bytes provenientes del canal serie, ensambla una instrucción de 32 bits y orienta la secuencia completa: lectura de operandos, ejecución en la ALU, write-back y/o envío por UART.
    \item \textbf{Un decodificador de instrucciones} que clasifica la instrucción (tipo R/I), extrae campos y selecciona el operador correspondiente.
    \item \textbf{Un banco de registros} que provee los operandos leídos por dirección y recibe el resultado bajo señal de escritura.
\end{itemize}

El objetivo principal es demostrar cómo la arquitectura UART puede extenderse más allá de la simple comunicación, integrándose con módulos de procesamiento para construir sistemas digitales completos. En particular, se implementa un mecanismo mediante el cual se reciben instrucciones por el puerto serie, se decodifican y procesan mediante la ALU con apoyo de un regfile, y los resultados se reenvían al transmisor UART.

El trabajo se organiza de la siguiente manera: primero se describen los módulos fundamentales en orden de ejecución (generador de baudios, receptor, FIFO, transmisor, ALU, interfaz, decoder y regfile). Posteriormente, se analiza la integración en el módulo top, se presentan las simulaciones y pruebas realizadas, y finalmente se discuten los resultados obtenidos. Cabe destacar que la migración desde una interfaz de tres bytes en orden fijo hacia una interfaz basada en comandos con decodificador y regfile constituye mejoras propuestas e implementadas en esta versión.
