\section{Unidad Aritmético-Lógica (ALU)}

La ALU es el bloque encargado de realizar operaciones aritméticas y lógicas sobre dos operandos de 8 bits.  
Se controla mediante un código de operación de 6 bits que determina qué función ejecutar.  

\subsection{Operaciones soportadas}
\begin{table}[H]
    \centering
    \begin{tabular}{|c|c|c|}
        \hline
        Código (binario) & Operación & Descripción \\ \hline
        100000 & ADD & Suma de \texttt{A + B} \\
        100010 & SUB & Resta \texttt{A - B} \\
        100100 & AND & AND bit a bit \\
        100101 & OR  & OR bit a bit \\
        100110 & XOR & XOR bit a bit \\
        000011 & SRA & Desplazamiento aritmético a derecha \\
        000010 & SRL & Desplazamiento lógico a derecha \\
        100111 & NOR & NOR bit a bit \\ \hline
    \end{tabular}
    \caption{Operaciones soportadas por la ALU.}
    \label{tab:alu-ops}
\end{table}

\subsection{Rol en el sistema}
La ALU recibe como entradas dos datos (\texttt{data\_a}, \texttt{data\_b}) y un código de operación (\texttt{op}), generando un resultado de 8 bits que se almacena en la FIFO de transmisión.  
Esto permite extender la comunicación UART más allá de la simple transferencia de bytes, agregando capacidad de procesamiento en hardware.
